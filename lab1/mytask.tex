\begin{problem}{1}{Захаревич Николай Сергеевич}{input.txt}{Консоль}

- Создать класс матрица (должен быть конструктор и деструктор)

- Выполнить перегрузку операторов +/-/*/+=/-=/*= для матрицы (поэлементно) и скалярного значения

- Выполнить перегрузку оператора () для индексации элементов матрицы

- Добавить метод транспонирования матрицы

- Добавить метод dot для произведения матриц

- Выполнить перегрузку operator<\!<(>\!>). Для >\!> данные для матрицы читаем из файла

\InputFile
Входной файл содержит число n \begin{math}(0 < n < 10^9)\end{math}- количество матриц, затем n матриц, заданных через размер
\begin{math}n_i, m_i (0 < n_i, m_i < 10^9) \end{math}, и \begin{math}n_i \cdot m_i \end{math} целых чисел. Затем идет число q - \begin{math}(0 < q < 10^9)\end{math}
количество запросов, относящихся к этим матрицам. Существует 5 типов запросов: a(i, j) - вывести заданный элемент матрицы a; a +=/-=/*= C - 
произвести операцию со всеми элементами матрицы a; a = b +/-/* C - присваивает a модифицированную b; a transpose - транспонирует a, a dot b - перемножает матрицы a и b

\OutputFile
Матрицы после исполнения всех запросов

\Example

\begin{example}
\exmp{
2
2 3
1 1 1
2 2 2
\begin{math}\end{math}
2 3
4 4 4
4 4 4
\begin{math}\end{math}
4
1 transpose
0 dot 1
1 += 3
1 *= 2

}{
12 12
24 24
\begin{math}\end{math}
14 14
14 14
14 14
}%
\end{example}

\Code

\printCode{codes/main.cpp}
\printCode{codes/matrix.h}
\printCode{codes/matrix.cpp}

\end{problem}
